%%%%%%%%%%%%%%%%%%%%%%%%%%%%%%%%%%%%%%%%%%%%%%%%%%%%%%%%%%%%%%%
%
% Welcome to Overleaf --- just edit your LaTeX on the left,
% and we'll compile it for you on the right. If you open the
% 'Share' menu, you can invite other users to edit at the same
% time. See www.overleaf.com/learn for more info. Enjoy!
%
%%%%%%%%%%%%%%%%%%%%%%%%%%%%%%%%%%%%%%%%%%%%%%%%%%%%%%%%%%%%%%%


% Inbuilt themes in beamer
\documentclass{beamer}
\usepackage[utf8]{inputenc}
\usepackage{amsmath}
\usepackage{amsfonts}
\usepackage{mathtools}
\usepackage{amssymb}
\providecommand{\pr}[1]{\ensuremath{\Pr\left(#1\right)}}


% Theme choice:
\usetheme{CambridgeUS}

% Title page details: 
\title{AI1110 Assignment-6} 
\author{Gollapudi Sasank CS21BTECH11019}
\date{\today}
\logo{\large \LaTeX{}}


\begin{document}

% Title page frame
\begin{frame}
    \titlepage 
\end{frame}

% Remove logo from the next slides
\logo{}


% Outline frame
\begin{frame}{Outline}
    \tableofcontents
\end{frame}


% Lists frame
\section{Question}
\begin{frame}{Question}

In a factory which manufactures bolts, machines A, B and
C manufacture respectively 25\% , 35\% and 40\% of the bolts.
Of their outputs, 5, 4 and 2 percent are respectively defective
bolts. A bolt is drawn at random from the product and is found
to be defective. What is the probability that it is manufactured
by the machine B?

\end{frame}


% Blocks frame
\section{Solution}
\begin{frame}{Solution}
Let  the random variable $X$ denote the following : \\
$X=0$ : the bolt is manufactured by machine A \\
$X=1$ : the bolt is manufactured by machine B \\
$X=2$ : the bolt is manufactured by machine C \\
A bolt must be manufactured from exactly one of the machines A,B,C.\\
Therefore $X=0,X=1,X=2$ are mutually exclusive and exhaustive events and hence, they represent a partition of the sample space.\\
Let the random variable $Y$ denote the following : \\
$Y=0$ : the bolt drawn at random is defective \\
$Y=1$ : the bolt drawn at random is not defective \\
\end{frame} 

\begin{frame}
Given that 
\begin{align}
\pr{X=0} &= 25\% = 0.25 \\
\pr{X=1} &= 35\% = 0.35 \\
\pr{X=2} &= 40\% = 0.4 
\end{align} 
And 
\begin{align}
\pr{Y=0|X=0} &= 5\% = 0.05 \\
\pr{Y=0|X=1} &= 4\% = 0.04 \\
\pr{Y=0|X=2} &= 2\% = 0.02
\end{align}
From Bayes Theorem , \\
\begin{footnotesize}
\begin{align}
\pr{X=1|Y=0} &= \frac{\pr{X=1}\pr{Y=0|X=1}}{\splitfrac{\pr{X=0}\pr{Y=0|X=0}+\pr{X=1}\pr{Y=0|X=1}}{+\pr{X=2}\pr{Y=0|X=2}}}
\end{align}
\end{footnotesize}

\end{frame}

\begin{frame}

\begin{align}
\Rightarrow \pr{X=1|Y=0} &= \frac{0.35 \times 0.04}{0.25 \times 0.05 + 0.35 \times 0.04 + 0.4 \times 0.02 } \\
\Rightarrow \pr{X=1|Y=0} &= \frac{0.014}{0.0125+0.014+0.008} \\
\Rightarrow \pr{X=1|Y=0} &= \frac{0.014}{0.0345} \\ 
\Rightarrow \pr{X=1|Y=0} &= \frac{28}{69} \\ 
\therefore  \pr{X=1|Y=0} &= \frac{28}{69} = 0.4058 
\end{align}

\end{frame}

\section{Graph}
\begin{frame}{Graph}
\begin{figure}[h]
\includegraphics[width=0.7\textwidth]{fig-1.png}
\caption{Probability Mass Function(PMF) graph}
\label{Fig-1}
\end{figure}
\end{frame}

\end{document}